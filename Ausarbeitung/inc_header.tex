\usepackage{ngerman}			% deutsches Tastaturlayout
\usepackage[ngerman]{babel}		% Neue deutsche Rechtschreibung
\usepackage{babelbib}			% Literaturverzeichnis mehrsprachig
\usepackage{pst-all,			% psTricks einbinden
	    graphicx}  			% Grafikpaket
\usepackage{xcolor}			% Farben verwenden, z.B. \color{black}
\usepackage[utf8]{inputenc}		% Textdateityp / Kodierung
\usepackage{pstricks}			% FH-Logo zeichnen lassen
% \usepackage{picins}			% Grafiken mit Textumfluss
\usepackage{float}			% zum Positionieren von Grafiken
\usepackage{fancyhdr}			% Kopfzeilen formatieren
\usepackage{helvet}			% Schriftart Helvetica
\usepackage{setspace}			% anderthalbzeiligen Zeilenabstand und so
%\usepackage{tikz}			% Zeichnen mit pgf und TikZ
\usepackage[titles]{tocloft} 		% ändert Verzeichnisse
\usepackage{listings}			% Programmcode listings
\usepackage[paper=a4paper,		% Ausgabegröße
	    left=30mm, 			% linker Seitenrand
	    right=25mm, 		% rechter Seitenrand
	    top=25mm,  			% oberer Seitenrand
	    includehead, 		% Kopfzeile bei der Berechnung der Body-Größe mit einbeziehen
	    bottom=25mm, 		% oberer Seitenrand
	    includefoot			% Fußzeile bei der Berechnung der Body-Größe mit einbeziehen
	    ]{geometry} 		% Layout der Seiten anpassen

%-------Hyperref und Hypercap als letztes Paket einbinden!-------
\usepackage[colorlinks=true,		% farbige Links aktivieren
	    linkcolor=black,		% Linkfarbe festlegen
	    %breaklinks=true,		% Links dürfen umgebrochen werden (Wichtig im Inhaltsverzeichnis)
 	    linktocpage=true,		% Abbildungsverzeichnis mit Bildern verlinken und Links nur auf die Zahlen setzen
 	    linkcolor=black,		% Dokument-interne Links
 	    citecolor=black, 		% Links zum Literaturverzeichnis
 	    urlcolor=blue,		% Farbe des verlinkten Textes, externe URLs
 	    filecolor=blue,		% Links auf lokale Dateien
%  	    frenchlinks=true,		% Links werden als smallcaps, anstatt farbig dargestellt.
	    pdfborder={0 0 0},		% Rahmen um Links auf 0 festlegen
	    ]{hyperref}			% anklickbare Links im Inhaltsverzeichnis
\usepackage[all]{hypcap}		% Erlaubt Links auf Abbildungen nicht nur unterhalb der caption zu setzen

\renewcommand{\familydefault}{\sfdefault}% Schriftart benutzen
% \columnsep 1cm			% 1 cm Abstand zwischen den Spalten
\setlength{\headheight}{34pt}		% Größe der Kopfzeile festlegen
\setlength{\topmargin}{1pt}

\DeclareGraphicsExtensions{.png,.jpg,.jpeg}

\input{./inc_color-definition.tex}
\lstdefinelanguage{Qt}{
	% classes:
	morekeywords=[1]{class, public, private, protected, slots, signals, this},
	% variables:
	morekeywords=[2]{int, char, double, float, void, bool},
	% Qt Classes
	morekeywords=[3]{QAbstractButton, QFlag, QMutex, QStyleOptionTab, QAbstractEventDispatcher, QFlags, QMutexLocker, QStyleOptionTabBarBase, QAbstractExtensionFactory, QFocusEvent, QNetworkAccessManager, QStyleOptionTabV2, QAbstractExtensionManager, QFocusFrame, QNetworkAddressEntry,  QStyleOptionTabWidgetFrame, QAbstractFileEngine, QFont,
	QNetworkCookie, QStyleOptionTitleBar, QAbstractFileEngineHandler, QFontComboBox, QNetworkCookieJar,
	QStyleOptionToolBar, QAbstractFileEngineIterator, QFontDatabase, QNetworkInterface, QStyleOptionToolBox, QAbstractFontEngine, QFontDialog, QNetworkProxy, QStyleOptionToolBoxV2, QAbstractFormBuilder, QFontEngineInfo, QNetworkReply, QStyleOptionToolButton, QAbstractGraphicsShapeItem, QFontEnginePlugin, QNetworkRequest, QStyleOptionViewItem, QAbstractItemDelegate, QFontInfo, QStyleOptionViewItemV2, QAbstractItemModel, QFontMetrics, QObject, QStyleOptionViewItemV4, QAbstractItemView, QFontMetricsF, QObjectCleanupHandler, QStylePainter, QAbstractListModel, QFormBuilder, QStylePlugin, QAbstractMessageHandler, QFormLayout, QPageSetupDialog, QSvgGenerator, QAbstractPrintDialog, QFrame, QPaintDevice, QSvgRenderer, QAbstractProxyModel, QFSFileEngine, QPaintEngine, QSvgWidget, QAbstractScrollArea, QFtp, QPaintEngineState, QSyntaxHighlighter, QAbstractSlider, QFuture, QPainter, QSysInfo, QAbstractSocket, QFutureIterator, QPainterPath, QSystemLocale, QAbstractSpinBox, QFutureSynchronizer, QPainterPathStroker, QSystemSemaphore, QAbstractTableModel, QFutureWatcher, QPaintEvent, QSystemTrayIcon, QAbstractTextDocumentLayout, QGenericArgument, QPair, QTabBar, QAbstractUriResolver, QGenericReturnArgument, QPalette, QTabletEvent, QAbstractXmlNodeModel, QGLColormap, QTableView, QAbstractXmlReceiver, QGLContext, QPen, QTableWidget, QAccessible, QGLFormat, QPersistentModelIndex, QTableWidgetItem, QAccessibleBridge, QGLFramebufferObject, QPicture, QTableWidgetSelectionRange, QAccessibleBridgePlugin, QGLPixelBuffer, QPictureFormatPlugin, QTabWidget, QAccessibleEvent, QGLWidget, QPictureIO, QTcpServer, QAccessibleInterface, QGradient, QPixmap, QTcpSocket, QAccessibleObject, QGraphicsEllipseItem, QPixmapCache, QTemporaryFile, QGraphicsGridLayout, QPlainTextDocumentLayout, QTestEventList, QAccessibleWidget, QGraphicsItem, QPlainTextEdit, QTextBlock, QAction, QGraphicsItemAnimation,
	QPlastiqueStyle, QTextBlockFormat, QActionEvent, QGraphicsItemGroup, QPluginLoader, QTextBlockGroup, QActionGroup, QGraphicsLayout, QPoint, QTextBlockUserData, QApplication, QGraphicsLayoutItem, QPointer, QTextBoundaryFinder, QAssistantClient, QGraphicsLinearLayout, QPointF, QTextBrowser, QAtomicInt, QGraphicsLineItem, QPolygon, QTextCharFormat, QAtomicPointer, QGraphicsPathItem, QPolygonF, QTextCodec, QGraphicsPixmapItem, QPrintDialog, QTextCodecPlugin, QAuthenticator, QGraphicsPolygonItem, QPrintEngine, QTextCursor, QAxAggregated, QGraphicsProxyWidget, QPrinter, QTextDecoder, QAxBase, QGraphicsRectItem, QPrinterInfo, QTextDocument, QAxBindable, QGraphicsScene, QPrintPreviewDialog, QTextDocumentFragment, QAxFactory, QGraphicsSceneContextMenuEvent, QPrintPreviewWidget, QTextEdit, QAxObject, QGraphicsSceneDragDropEvent, QProcess, QTextEncoder, QAxScript, QGraphicsSceneEvent, QProgressBar,  QTextFormat, QAxScriptEngine, QGraphicsSceneHelpEvent, QProgressDialog, QTextFragment, QAxScriptManager, QGraphicsSceneHoverEvent, QProxyModel, QTextFrame, QAxWidget, QGraphicsSceneMouseEvent, QProxyScreen, QTextFrameFormat, QBasicTimer, QGraphicsSceneMoveEvent, QPushButton, QTextImageFormat, QBitArray, QGraphicsSceneResizeEven, QQueue, QTextInlineObject, QBitmap, QGraphicsSceneWheelEvent, QRadialGradient, QTextItem, QBoxLayout, QGraphicsSimpleTextItem, QRadioButton, QTextLayout, QBrush, QGraphicsSvgItem, QRasterPaintEngine, QTextLength, QBuffer, QGraphicsTextItem, QReadLocker, QTextLine, QButtonGroup, QGraphicsView, QReadWriteLock,
	QTextList, QByteArray, QGraphicsWidget, QHelpEvent, QAccessiblePlugin, QRect, QTextListFormat, QByteArrayMatcher, QGridLayout, QRectF, QTextObject, QCache, QGroupBox, QRegExp, QTextOption, QCalendarWidget, QHash, QRegExpValidator, QTextStream, QCDEStyle, QHashIterator, QRegion, QTextTable, QChar, QHBoxLayout, QResizeEvent, QTextTableCell, QCheckBox, QHeaderView, QResource,
	QTextTableCellFormat, QChildEvent, QHelpContentItem, QRubberBand, QTextTableFormat, QCleanlooksStyle, QHelpContentModel, QRunnable, QThread, QClipboard, QHelpContentWidget, QScreen, QThreadPool, QCloseEvent, QHelpEngine, QScreenCursor, QThreadStorage, QColor, QHelpEngineCore, QScreenDriverFactory, QTime, QColorDialog, QScreenDriverPlugin, QTimeEdit, QColormap, QHelpIndexModel, QScriptable, QTimeLine, QColumnView, QHelpIndexWidget, QScriptClass, QTimer, QComboBox, QHelpSearchEngine, QScriptClassPropertyIterator, QTimerEvent, QCommandLinkButton, QHelpSearchQuery, QScriptContext, QToolBar, QCommonStyle, QHelpSearchQueryWidget, QScriptContextInfo, QToolBox, QCompleter, QHelpSearchResultWidget, QScriptEngine, QToolButton, QConicalGradient, QHideEvent, QScriptEngineAgent, QToolTip, QContextMenuEvent,
	QScriptExtensionPlugin, QTransform, QCopChannel, QHostInfo, QScriptString, QTranslator, QCoreApplication, QHoverEvent, QScriptValue, QTreeView, QCryptographicHash, QHttp, QScriptValueIterator, QTreeWidget, QCursor, QHttpHeader, QScrollArea, QTreeWidgetItem, QCustomRasterPaintDevice, QHttpRequestHeader, QScrollBar, QTreeWidgetItemIterator, QDataStream, QHttpResponseHeader, QUdpSocket, QDataWidgetMapper, QIcon, QSemaphore, QUiLoader, QDate, QIconDragEvent, QSessionManager, QUndoCommand, QDateEdit, QIconEngine, QSet, QUndoGroup, QDateTime, QIconEnginePlugin, QSetIterator, QUndoStack, QDateTimeEdit, QIconEnginePluginV2, QSettings,
	QUndoView, QDBusAbstractAdaptor, QIconEngineV2, QSharedData, QDBusAbstractInterface, QImage, QSharedDataPointer, QUrl, QDBusArgument, QImageIOHandler, QSharedMemory, QUrlInfo, QDBusConnection, QImageIOPlugin, QShortcut, QUuid, QDBusConnectionInterface, QImageReader, QShortcutEvent,
	QValidator, QDBusContext, QImageWriter, QShowEvent, QVariant, QDBusError, QInputContext, QSignalMapper, QVarLengthArray, QDBusInterface, QInputContextFactory, QSignalSpy, QVBoxLayout, QDBusMessage, QInputContextPlugin, QSimpleXmlNodeModel, QVector, QDBusObjectPath, QInputDialog, QSize, QVectorIterator, QDBusReply, QInputEvent, QSizeF, QDBusServer, QInputMethodEvent, QSizeGrip, QDBusSignature, QIntValidator, QSizePolicy, QHostAddress, QDBusVariant, QIODevice, QSlider,  QWaitCondition, QDecoration, QItemDelegate, QSocketNotifier, QWebFrame, QDecorationDefault, QItemEditorCreator, QSortFilterProxyModel, QWebHistory, QDecorationFactory, QItemEditorCreatorBase, QSound, QWebHistoryInterface, QDecorationPlugin, QItemEditorFactory, QSourceLocation, QWebHistoryItem, QDesignerActionEditorInterface, QItemSelection, QSpacerItem, QWebHitTestResult, QDesignerContainerExtension, QItemSelectionModel, QSpinBox, QWebPage, QDesignerCustomWidgetCollectionInterface, QItemSelectionRange, QSplashScreen, QWebPluginFactory, QDesignerCustomWidgetInterface, QKbdDriverFactory, QSplitter, QWebSettings, QDesignerDynamicPropertySheetExtension, QKbdDriverPlugin, QSplitterHandle, QWebView, QDesignerFormEditorInterface, QKeyEvent, QSqlDatabase, QWhatsThis, QDesignerFormWindowCursorInterface, QKeySequence, QSqlDriver, QWhatsThisClickedEvent, QDesignerFormWindowInterface, QLabel, QSqlDriverCreator, QWheelEvent, QDesignerFormWindowManagerInterface, QLatin1Char, QSqlDriverCreatorBase, QWidget, QDesignerMemberSheetExtension, QLatin1String, QSqlDriverPlugin, QWidgetAction, QDesignerObjectInspectorInterface, QLayout, QSqlError, QWidgetItem, QDesignerPropertyEditorInterface, QLayoutItem, QSqlField, QWindowsMime, QDesignerPropertySheetExtension, QLCDNumber, QSqlIndex, QWindowsStyle, QDesignerTaskMenuExtension, QLibrary, QSqlQuery, QWindowStateChangeEvent, QDesignerWidgetBoxInterface, QLibraryInfo, QSqlQueryModel, QWindowsVistaStyle, QDesktopServices, QLine, QSqlRecord, QWindowsXPStyle, QDesktopWidget, QLinearGradient, QSqlRelation, QWizard, QDial, QLineEdit, QSqlRelationalDelegate, QWizardPage, QDialog, QLineF, QWorkspace, QDialogButtonBox, QLinkedList, QSqlResult, QWriteLocker, QDir, QLinkedListIterator, QSqlTableModel, QWSCalibratedMouseHandler, QDirectPainter, QList,
	QSslCertificate, QWSClient, QDirIterator, QListIterator, QSslCipher, QWSEmbedWidget, QDirModel, QListView, QSslConfiguration, QWSEvent, QDockWidget, QListWidget, QSslError, QWSGLWindowSurface, QDomAttr, QListWidgetItem, QSslKey, QWSInputMethod, QDomCDATASection, QLocale, QSslSocket, QWSKeyboardHandler, QDomCharacterData, QLocalServer, QStack, QWSMouseHandler, QDomComment, QLocalSocket, QStackedLayout, QWSPointerCalibrationData, QDomDocument, QMacPasteboardMime, QStackedWidget, QWSScreenSaver, QDomDocumentFragment, QMacStyle, QStandardItem, QWSServer, QDomDocumentType, QMainWindow, QStandardItemEditorCreator, QWSWindow, QDomElement, QMap,
	QStandardItemModel, QX11EmbedContainer, QDomEntity, QMapIterator, QStatusBar, QX11EmbedWidget, QDomEntityReference, QMatrix, QStatusTipEvent, QX11Info, QDomImplementation, QMdiArea, QString, QXmlAttributes, QDomNamedNodeMap, QMdiSubWindow, QStringList, QXmlContentHandler, QDomNode,
	QStringListModel, QXmlDeclHandler, QDomNodeList, QStringMatcher, QXmlDefaultHandler, QDomNotation,  QStringRef, QSqlRelationalTableModel, QXmlDTDHandler, QDomProcessingInstruction, QStyle, QXmlEntityResolver, QDomText, QMenu, QStyledItemDelegate, QXmlErrorHandler, QDoubleSpinBox,
	QMenuBar, QStyleFactory, QXmlFormatter, QDoubleValidator, QMessageBox, QStyleHintReturn, QXmlInputSource, QDrag, QMetaClassInfo, QStyleHintReturnMask, QXmlItem, QDragEnterEvent, QMetaEnum, QStyleHintReturnVariant, QXmlLexicalHandler, QDragLeaveEvent, QMetaMethod, QStyleOption, QXmlLocator, QDragMoveEvent, QStyleOptionButton, QXmlName, QDropEvent, QMetaProperty, QStyleOptionComboBox, QXmlNamePool, QDynamicPropertyChangeEvent, QMetaType, QStyleOptionComplex,
	QXmlNamespaceSupport, QMimeData, QStyleOptionDockWidget, QXmlNodeModelIndex, QMimeSource, QStyleOptionFocusRect, QMetaObject, QXmlParseException, QModelIndex, QStyleOptionFrame, QXmlQuery, QErrorMessage, QMotifStyle, QStyleOptionFrameV2, QXmlReader, QEvent, QMouseDriverFactory, QStyleOptionGraphicsItem, QXmlResultItems, QEventLoop, QMouseDriverPlugin, QStyleOptionGroupBox, QXmlSerializer, QMouseEvent, QStyleOptionHeader, QXmlSimpleReader, QExplicitlySharedDataPointer, QMoveEvent, QStyleOptionMenuItem, QXmlStreamAttribute, QExtensionFactory, QMovie,
	QStyleOptionProgressBar, QXmlStreamAttributes, QExtensionManager, QMultiHash, QStyleOptionProgressBarV2, QXmlStreamEntityDeclaration,QFile, QMultiMap,
	QStyleOptionQ3DockWindow, QXmlStreamEntityResolver, QFileDialog, QMutableHashIterator, QStyleOptionQ3ListView, QXmlStreamNamespaceDeclaration, QFileIconProvider,
	QMutableLinkedListIterator, QStyleOptionQ3ListViewItem, QXmlStreamNotationDeclaration, QFileInfo, QMutableListIterator, QStyleOptionRubberBand, QXmlStreamReader, QFileOpenEvent, QMutableMapIterator, QStyleOptionSizeGrip, QXmlStreamWriter, QFileSystemModel, QMutableSetIterator, QStyleOptionSlider, QFileSystemWatcher, QMutableVectorIterator, QStyleOptionSpinBox
	}, % KDE Classes
	morekeywords=[4]{KApplication, KXMLGuiWindow, KPushButton, KMenu, KTabBar, KTextEditor, KAboutData, ki18n, KGuiItem, KMessageBox, KCmdLineArgs, KActionCollection}, % Qt Macros
	morekeywords=[5]{Q_OBJECT, Q_PROPERTY, READ, WRITE, SLOT, SIGNAL, connect, disconnect},
	morekeywords=[6]{if, for, else, while, do, case, switch, foreach, forever, new},
	% pre compiler directives
	morecomment=[s]{/*}{*/},%
	morecomment=[l]//,% nonstandard
	morestring=[b]",%
	morestring=[b]',
	sensitive=true,
	moredelim=*[directive]\#,
	moredirectives={define,elif,else,endif,error,if,ifdef,ifndef,line,
	include,pragma,undef,warning, emit},
	basicstyle=\ttfamily\scriptsize,
  }[keywords,comments,strings,directives]%

\lstset{
  language=Qt,			% choose the language of the code
  showspaces=false,		% show spaces within strings adding particular underscores
  showtabs=false,		% show tabs within strings adding particular underscores
  frame=single,
  showstringspaces=false,	% underline spaces within strings
  keywordstyle=[1]{\color{DarkSkyBlue}},
  keywordstyle=[2]{\color{DarkScarletRed}},
  keywordstyle=[3]{\color{DarkGreen}},
  keywordstyle=[4]{\color{blue}},
  keywordstyle=[5]{\color{Chocolate}},
  keywordstyle=[6]{\bfseries},
  commentstyle={\color{Aluminium4}},
  stringstyle={\color{Chocolate}},
  tabsize=4,
  breaklines=true,
  breakatwhitespace=true,
  prebreak=\mbox{{\color{blue}\tiny$\searrow$}},
  postbreak=\mbox{{\color{blue}\tiny$\rightarrow$}},
  basicstyle={			% the size of the fonts that are used for the code
    %\ttfamily
    \scriptsize
    },
  xleftmargin=5pt,
  xrightmargin=5pt,
  rulecolor=\color{black!30},
  backgroundcolor=\color{white},% choose the background color. You must add \usepackage{color}
  captionpos=b,
  framesep=10pt,
  framexleftmargin=18pt,
  numbers=none,			% where to put the line-numbers
  numberstyle={\tiny},		% the size of the fonts that are used for the line-numbers
  stepnumber=1,			% the step between two line-numbers. If it's 1 each line will be numbered
  numbersep=15pt,		% how far the line-numbers are from the code
  escapeinside={\%*}{*)}%,      % if you want to add a comment within your code
%morecomment=[is]{///}{</summary>}
%morecomment=[l][keywordstyle]{//}
}
\lstloadlanguages{Qt,XML}


%-------SEITENLAYOUT FÜR PAKET fancyhdr-----------
\pagestyle{fancy}			% eigenen Seitenstil von fancyhdr aktivieren
\fancyhf{}%Löschung der Vorbelegung
%-------Kopfzeile-------
\fancyhead[L]{\resizebox*{1cm}{!}{%LaTeX with PSTricks extensions
%%Creator: inkscape 0.46
%%Please note this file requires PSTricks extensions
\psset{xunit=.5pt,yunit=.5pt,runit=.5pt}
\begin{pspicture}(800,600)
{
\newrgbcolor{curcolor}{0 0 0}
\pscustom[linestyle=none,fillstyle=solid,fillcolor=curcolor]
{
\newpath
\moveto(78.34211,462.87504)
\curveto(79.63005,350.25949)(80.91799,237.64387)(82.20581,125.02832)
\curveto(101.99252,125.02832)(121.77929,125.02832)(141.56606,125.02832)
\curveto(141.56606,170.84037)(141.56606,216.6526)(141.56606,262.46477)
\curveto(174.46576,262.46477)(207.36565,262.46477)(240.26553,262.46477)
\curveto(240.26553,281.04486)(240.26553,299.62506)(240.26553,318.2052)
\curveto(206.89726,318.2052)(173.52917,318.2052)(140.16108,318.2052)
\curveto(140.51231,349.26668)(140.86348,380.3281)(141.21471,411.38964)
\curveto(175.75376,411.38964)(210.29275,411.38964)(244.83163,411.38964)
\curveto(244.83163,429.54421)(244.83163,447.69896)(244.83163,465.85364)
\curveto(189.3352,464.86072)(133.83866,463.86797)(78.34211,462.87504)
\closepath
}
}
{
\newrgbcolor{curcolor}{0.94117647 0.64705884 0.07450981}
\pscustom[linestyle=none,fillstyle=solid,fillcolor=curcolor]
{
\newpath
\moveto(526.66082157,117.16311448)
\lineto(457.3157015,117.16311448)
\lineto(457.3157015,329.96798209)
\curveto(457.3155111,351.18394332)(455.14092763,366.61388808)(450.79194455,376.25786265)
\curveto(446.60367399,386.11562382)(439.27452229,391.04463395)(428.80446746,391.04490783)
\curveto(410.76331677,391.04463395)(391.27260564,378.29349905)(370.33227562,352.79146486)
\lineto(370.33227562,117.16311448)
\lineto(300.98715555,117.16311448)
\lineto(300.98715555,595.17042283)
\lineto(370.33227562,595.17042283)
\lineto(370.33227562,413.54693311)
\curveto(381.60779021,427.90505754)(394.73583117,439.79897329)(409.71643789,449.22871604)
\curveto(424.69675901,458.65779466)(438.95236178,463.37250001)(452.48328895,463.37284622)
\curveto(501.93474235,463.37250001)(526.66056183,423.1903522)(526.66082157,342.82628224)
\lineto(526.66082157,117.16311448)
}
}
{
\newrgbcolor{curcolor}{0 0 0}
\pscustom[linestyle=none,fillstyle=solid,fillcolor=curcolor]
{
\newpath
\moveto(198.57924413,47.99697332)
\lineto(208.37494229,36.74149818)
\curveto(215.31720799,28.95758824)(219.3240208,24.70703803)(220.39539273,23.98983482)
\curveto(221.43992962,23.3026367)(222.79517023,22.81161963)(224.46111863,22.51678211)
\curveto(226.13027742,22.27875677)(228.29412656,23.21039495)(230.95267256,25.31169943)
\curveto(232.35866516,26.45775251)(233.32971575,26.73025619)(233.86582723,26.12921129)
\curveto(234.45548912,25.46800652)(234.26947751,24.70855127)(233.30779185,23.85084327)
\curveto(232.94712071,23.52916964)(231.93342314,22.75997352)(230.26669609,21.5432526)
\curveto(226.87305887,19.05613714)(223.97411038,16.74042519)(221.56984193,14.59610979)
\curveto(219.52618297,12.77341425)(216.79333773,10.12020991)(213.371298,6.6364888)
\lineto(210.96741483,4.24969863)
\curveto(210.00569476,3.39196204)(209.25679647,3.26363163)(208.72071769,3.86470701)
\curveto(208.18463135,4.46577417)(208.62284881,5.39620682)(210.0353714,6.65600772)
\curveto(211.95879437,8.37146502)(212.85878399,10.14541905)(212.73534296,11.97787514)
\curveto(212.61188634,13.81030679)(210.70027277,17.04268216)(207.00049654,21.67501094)
\curveto(203.30396561,26.36415359)(198.2124017,32.34521706)(191.72578952,39.6182193)
\curveto(185.61446156,46.47039193)(180.68290634,51.75779008)(176.93110907,55.4804296)
\curveto(173.15256368,59.23301125)(170.48839063,61.52439703)(168.93858193,62.35459381)
\curveto(167.41889879,63.21146953)(165.99542746,63.50673412)(164.66816368,63.24038847)
\curveto(163.3141763,63.00396943)(161.35298283,61.87535837)(158.78457739,59.85455192)
\curveto(157.76930125,59.05686686)(157.02040296,58.92853645)(156.53788026,59.4695603)
\curveto(156.10906261,59.95035761)(156.28857535,60.59609747)(157.07641901,61.40678183)
\curveto(164.0537752,68.70882344)(177.98604271,81.67430953)(198.87326332,100.30327899)
\curveto(202.44964345,103.49286968)(204.44170656,105.1616288)(204.84945862,105.30956138)
\curveto(205.26046029,105.51414054)(205.66008884,105.51982334)(206.04834549,105.32660978)
\curveto(206.43984935,105.19004278)(208.69505728,103.47820262)(212.81397607,100.19108417)
\curveto(216.96293265,96.93056898)(219.11782727,95.21020091)(219.27866638,95.0299748)
\curveto(219.84153873,94.3987547)(219.8525,93.84195513)(219.31155023,93.35957442)
\curveto(218.92083953,93.01102716)(218.2771287,92.9765112)(217.38041583,93.25602643)
\curveto(206.57583957,96.83967397)(196.41006784,94.3830902)(186.88307014,85.88626777)
\curveto(183.60724976,82.9645317)(181.08153003,80.44209856)(179.30590336,78.31896078)
\curveto(177.53357738,76.25252358)(176.66973587,74.86154661)(176.71437624,74.1460257)
\curveto(176.73226893,73.46040958)(178.1821252,71.38103089)(181.0639494,67.90788341)
\curveto(183.97587491,64.46140376)(189.17100158,58.54572589)(196.64934499,50.16083206)
\curveto(199.13080131,52.15809351)(200.82232878,53.55881135)(201.72393247,54.3629898)
\curveto(206.05164884,58.22272962)(208.23947929,61.8197761)(208.2874304,65.15414004)
\curveto(208.33862424,68.54523088)(207.52476696,71.27278278)(205.84585611,73.33680394)
\curveto(204.7265807,74.71269542)(204.48250587,75.68212129)(205.11363088,76.24508446)
\curveto(205.7748091,76.83469879)(206.50746022,76.67874074)(207.31158643,75.77720985)
\curveto(207.57962534,75.4765982)(207.88462337,75.07412453)(208.22658146,74.56978763)
\curveto(209.57084264,72.63897723)(213.44606817,68.0822086)(219.85226966,60.89946804)
\lineto(224.51619259,55.67014275)
\curveto(225.64194638,54.40783717)(225.82916353,53.44166074)(225.07784458,52.77161055)
\curveto(224.4466996,52.20867085)(223.70064642,52.37965568)(222.83968279,53.28456555)
\curveto(220.33713268,55.90892036)(217.51050272,57.35393964)(214.35978442,57.61962771)
\curveto(211.18224314,57.91525363)(207.20421891,55.93216735)(202.42569979,51.67036291)
\curveto(201.64430965,50.97340368)(200.36215904,49.7489417)(198.57924413,47.99697332)
}
}
{
\newrgbcolor{curcolor}{0 0 0}
\pscustom[linestyle=none,fillstyle=solid,fillcolor=curcolor]
{
\newpath
\moveto(315.80856681,98.24090742)
\curveto(314.38949728,96.867351)(313.57480163,96.08678186)(313.36447741,95.89919764)
\curveto(312.25244484,94.90740023)(310.18524782,93.17163078)(307.16288017,90.69188409)
\curveto(304.16722759,88.18200491)(298.07122834,82.82605347)(288.87486413,74.62401372)
\lineto(277.42444496,64.41163076)
\curveto(277.00368028,64.03635978)(275.5245568,62.60924213)(272.98707006,60.13027352)
\curveto(270.47636676,57.62123354)(268.93551143,56.11207878)(268.36449945,55.60280473)
\curveto(266.08042289,53.56568663)(264.6703462,52.8476654)(264.13426513,53.4487389)
\curveto(263.70539759,53.92959369)(263.92836596,54.5871109)(264.8031709,55.4212925)
\curveto(267.55753963,58.20161009)(269.03667055,60.62698678)(269.24056809,62.69742984)
\curveto(269.4176483,64.79790352)(268.83203043,67.02754259)(267.48371272,69.38635374)
\curveto(266.10858119,71.77518213)(261.4942131,77.37245246)(253.64059462,86.17818154)
\lineto(239.29184391,102.35716755)
\curveto(234.62794899,107.58643647)(231.13081894,111.02351142)(228.80044326,112.66840271)
\curveto(226.50018881,114.33997523)(224.13125806,115.06005522)(221.69364389,114.82864485)
\curveto(219.25936105,114.65396495)(217.05855515,113.82426481)(215.09121959,112.33954194)
\curveto(214.04914072,111.57190942)(213.28684036,111.45860579)(212.80431622,111.99963071)
\curveto(212.32189175,112.54053408)(212.51643127,113.19967601)(213.38793536,113.97705847)
\curveto(213.59835956,114.16462462)(214.00245795,114.4980515)(214.60023175,114.97734012)
\curveto(219.62003354,118.96865064)(225.33061494,123.8189774)(231.73199308,129.52833495)
\lineto(239.5455231,136.73987298)
\curveto(240.3569924,137.46349892)(241.03075679,137.52481901)(241.56681828,136.92383344)
\curveto(242.04931365,136.38278359)(241.85477413,135.72364167)(240.98319914,134.94640568)
\curveto(238.84941778,133.04322887)(237.43648854,131.02763337)(236.74440717,128.89961312)
\curveto(236.08243116,126.79824166)(236.00687807,124.97717081)(236.51774766,123.43639512)
\curveto(237.03191894,121.9523327)(238.7096103,119.61749873)(241.55082677,116.4318862)
\lineto(260.20773144,94.78721167)
\lineto(270.9123375,82.96640106)
\curveto(274.20923697,79.269781)(276.50383502,76.99951655)(277.79613855,76.15560091)
\curveto(279.11846532,75.33843954)(280.67758201,74.92133597)(282.47349328,74.90428897)
\curveto(284.24256075,74.91724817)(286.43443442,76.08971917)(289.04912086,78.4217055)
\curveto(289.2594694,78.6093072)(291.86885095,81.09843734)(296.87727336,85.88910339)
\curveto(300.48529889,90.13225197)(302.38601064,95.44274846)(302.5794143,101.82060881)
\curveto(302.77273807,108.19836797)(302.19079625,113.2371843)(300.83358709,116.9370729)
\curveto(300.46723077,118.01324525)(300.52450021,118.7657973)(301.0053956,119.19473132)
\curveto(301.51626781,119.65033352)(302.09337264,119.51752585)(302.73671181,118.7963079)
\curveto(303.48718729,117.95474054)(305.74522548,114.54569755)(309.51083314,108.56916869)
\curveto(313.24954293,102.6225787)(315.34878539,99.17982838)(315.80856681,98.24090742)
}
}
{
\newrgbcolor{curcolor}{0 0 0}
\pscustom[linestyle=none,fillstyle=solid,fillcolor=curcolor]
{
\newpath
\moveto(318.7736937,206.26862441)
\curveto(322.64731004,203.67984061)(327.71208469,199.72534878)(333.96803284,194.40513704)
\curveto(334.03218352,194.03059537)(333.97410543,193.76295842)(333.7937984,193.60222537)
\curveto(333.58340911,193.41450475)(333.25079087,193.33368805)(332.79594269,193.35977501)
\curveto(327.74697525,194.84614426)(323.49849124,195.61659794)(320.05047793,195.67113835)
\curveto(316.60246339,195.72549119)(313.59509286,195.25562517)(311.02835731,194.26153889)
\curveto(308.46488801,193.32412554)(304.85399799,190.77814513)(300.19567639,186.62359004)
\curveto(295.80787535,182.71010262)(292.56452385,179.35877055)(290.46561215,176.56958378)
\curveto(290.49325389,176.05454044)(290.868916,175.391333)(291.59259961,174.57995948)
\curveto(291.78025636,174.36951184)(292.08525439,173.96703817)(292.50759463,173.37253726)
\curveto(294.35141617,170.7001394)(300.01764411,164.04449547)(309.50629547,153.40558549)
\curveto(309.90836733,152.95473051)(310.50130758,152.35040981)(311.285118,151.59262156)
\curveto(313.64960937,153.53952488)(315.25260087,154.88825919)(316.0940973,155.63882853)
\curveto(319.88084966,159.01609489)(322.11984758,162.01126146)(322.81109777,164.62433722)
\curveto(323.50559066,167.29414519)(322.8221077,171.05528817)(320.76064682,175.90777745)
\curveto(320.40813376,176.72645715)(320.42722605,177.31006085)(320.81792376,177.65859029)
\curveto(321.38894008,178.16779295)(322.06310879,177.98665549)(322.84043191,177.11517738)
\lineto(324.26835963,175.15113683)
\curveto(326.1454524,172.56239902)(328.20977457,170.00581708)(330.46133234,167.48138333)
\curveto(334.10668272,163.39403309)(337.4653223,159.77948188)(340.53726116,156.63771884)
\curveto(341.44858042,155.61584293)(341.61874174,154.85029313)(341.04774564,154.34106716)
\curveto(340.62697414,153.96575691)(340.01981575,154.07176045)(339.22626865,154.65907812)
\curveto(335.43223693,157.64245915)(331.83152926,159.01763043)(328.42413483,158.78459606)
\curveto(325.01672243,158.55144789)(319.94702161,155.43283993)(313.21501714,149.42876282)
\curveto(315.08156402,147.15439275)(318.19937077,143.56786605)(322.56844675,138.66917198)
\curveto(328.22411462,132.32783328)(332.04126949,128.50168672)(334.01992282,127.19072082)
\curveto(335.99854699,125.87970431)(338.3000631,125.47721578)(340.92447806,125.98325401)
\curveto(343.52204592,126.5192976)(346.86448291,128.61001215)(350.95179907,132.25540392)
\curveto(362.64260378,142.68217932)(366.59429509,153.43719296)(362.80688485,164.52047709)
\curveto(362.34305515,165.88743984)(362.33656371,166.77198349)(362.78741052,167.1741107)
\curveto(363.26822419,167.60291146)(363.8034981,167.48675528)(364.39323387,166.82564183)
\curveto(364.71483845,166.46493412)(364.98815821,166.00722748)(365.21319395,165.45252051)
\curveto(365.54210523,164.72061447)(367.00942164,162.1981516)(369.61514759,157.88512434)
\curveto(370.16173693,156.96965505)(371.6814412,154.11619866)(374.17426497,149.32474661)
\lineto(355.24537515,132.52341574)
\curveto(352.54051862,130.11101359)(348.45120493,126.67968089)(342.97742182,122.22940735)
\curveto(337.56044933,117.77584369)(332.80995725,113.75482018)(328.72593133,110.16632475)
\curveto(324.69874452,106.57456332)(322.52798475,104.77340828)(322.2136455,104.76285421)
\curveto(321.89930071,104.75229438)(321.66171731,104.83717659)(321.50089461,105.01750111)
\curveto(320.96480999,105.61856925)(321.19265251,106.36137299)(322.18442285,107.24591458)
\curveto(325.52036566,110.22116742)(326.92355403,113.06444888)(326.39399216,115.7757675)
\curveto(325.86441638,118.48705834)(322.54396138,123.26881695)(316.4326172,130.12105768)
\lineto(308.14525923,139.32233149)
\lineto(300.56090383,148.09838514)
\curveto(299.00628329,149.84144562)(295.96604381,153.03849638)(291.44017626,157.68954702)
\curveto(286.91761415,162.39734375)(284.25791094,165.01647361)(283.46105863,165.54694447)
\curveto(282.69433185,166.10409213)(281.62540874,166.36483332)(280.25428608,166.32916884)
\curveto(278.88649124,166.35023316)(276.58699674,165.54043396)(273.35579568,163.8997688)
\curveto(272.38682021,163.41317441)(271.70127848,163.39530994)(271.29916842,163.84617536)
\curveto(270.81674226,164.38707754)(271.10144247,165.12663185)(272.15326991,166.06484051)
\lineto(281.79193811,173.77102757)
\curveto(285.17178126,176.51553368)(293.21801271,183.55689574)(305.93065659,194.89513487)
\curveto(314.28556077,202.34658943)(318.56656886,206.13774882)(318.7736937,206.26862441)
}
}
{
\newrgbcolor{curcolor}{0 0 0}
\pscustom[linestyle=none,fillstyle=solid,fillcolor=curcolor]
{
\newpath
\moveto(344.22340309,229.37137101)
\curveto(348.19778707,228.54520659)(354.68727574,227.56111367)(363.69188857,226.41908929)
\lineto(393.03412562,222.56025872)
\lineto(421.23230451,218.89515923)
\curveto(416.68125178,224.60283254)(410.53254953,231.79944061)(402.78617932,240.48500505)
\curveto(396.96966079,247.00654378)(392.9478212,251.24369192)(390.72064849,253.19646218)
\curveto(388.46665025,255.17909146)(385.79799994,256.14447412)(382.71468954,256.09261305)
\curveto(379.66142734,256.06735576)(376.84249434,254.9021997)(374.25788211,252.59714137)
\curveto(373.05574533,251.52487958)(372.22683897,251.24425231)(371.77116054,251.75525873)
\curveto(371.50312752,252.05569853)(371.43571288,252.37328998)(371.56891642,252.70803405)
\curveto(371.70538631,253.09944223)(373.61338327,255.0439619)(377.29291303,258.5415989)
\curveto(380.94564414,262.06908422)(384.06431156,264.98545569)(386.64892465,267.29072206)
\lineto(392.77498359,272.67348301)
\lineto(393.76675217,273.55802011)
\curveto(394.69839891,274.38883044)(395.40546629,274.53381235)(395.88795646,273.99296628)
\curveto(396.31680664,273.51199119)(396.16572275,272.86462661)(395.43470432,272.0508706)
\curveto(392.6941244,269.01294835)(391.21052366,266.25982757)(390.98389762,263.79150001)
\curveto(390.76049912,261.37981441)(391.5589253,258.88131799)(393.37917855,256.29600325)
\curveto(395.22945867,253.73729189)(402.14532801,245.74101495)(414.1268073,232.30714842)
\lineto(434.90489395,208.10263995)
\curveto(435.97700419,206.90044523)(436.21255112,206.03133268)(435.61153543,205.49529969)
\curveto(435.46121203,205.36122948)(435.19722883,205.23370771)(434.81958504,205.112734)
\curveto(434.4450811,205.04851961)(431.75408906,205.37344087)(426.74660084,206.08749878)
\curveto(421.71221506,206.83150991)(415.59757987,207.63731049)(408.4026769,208.50490293)
\curveto(401.23777076,209.39919433)(395.50773733,210.19727564)(391.21255945,210.89914926)
\curveto(381.88302014,212.34497422)(376.06932489,213.17635853)(373.77145629,213.39330469)
\curveto(365.28586012,214.13489745)(360.06267177,214.81844731)(358.10187556,215.4439563)
\curveto(356.78686307,215.88983351)(355.91897091,215.92517298)(355.49819649,215.54997482)
\curveto(355.16763182,215.25506978)(355.52501772,214.52160259)(356.57035525,213.34957104)
\lineto(362.89248201,206.44252298)
\curveto(370.79971689,197.57667121)(377.35251753,190.71349001)(382.55090359,185.8529588)
\curveto(387.7492846,180.99236773)(391.01939564,178.32409201)(392.36124652,177.84812365)
\curveto(393.70632217,177.42896912)(395.44048176,177.32986133)(397.5637305,177.55079999)
\curveto(399.71699773,177.79849952)(401.52995242,178.57906123)(403.00260001,179.89248747)
\curveto(404.98611223,181.66153595)(406.29952936,182.18542941)(406.94285535,181.46416942)
\curveto(407.42530273,180.92317974)(407.11054896,180.1568213)(405.99859308,179.16509181)
\lineto(384.27471905,159.87097455)
\curveto(383.25289364,158.95963025)(382.51414797,158.75941536)(382.05847983,159.27032929)
\curveto(381.52239524,159.87139721)(381.75023776,160.61420096)(382.74200807,161.49874276)
\curveto(386.43859367,164.79564492)(388.17927443,167.80502963)(387.96405557,170.52690591)
\curveto(387.7220167,173.27880655)(385.22883591,177.31450454)(380.48450572,182.63401201)
\lineto(366.01147512,198.13557919)
\curveto(357.67540535,207.48219525)(352.51479464,212.75416229)(350.52962753,213.95149614)
\curveto(348.51771528,215.1787675)(345.55137444,216.17542411)(341.63059613,216.94146898)
\curveto(337.71314404,217.76424949)(334.47834274,217.30738122)(331.92618253,215.57086278)
\curveto(330.79719296,214.77967703)(330.04504522,214.59448893)(329.66973707,215.01529791)
\curveto(329.160509,215.58625653)(329.56704818,216.4614562)(330.88935584,217.64089955)
\lineto(344.22340309,229.37137101)
}
}
{
\newrgbcolor{curcolor}{0 0 0}
\pscustom[linestyle=none,fillstyle=solid,fillcolor=curcolor]
{
\newpath
\moveto(435.33449018,305.45126718)
\curveto(435.42223887,306.98633708)(435.21268743,308.31031116)(434.70583523,309.42319341)
\curveto(434.30297923,310.35882021)(434.32694568,311.02771044)(434.77773464,311.42986611)
\curveto(435.01817764,311.64420539)(435.42268035,311.73517472)(435.99124399,311.70277439)
\curveto(437.16658082,310.80838097)(439.41041573,308.89753889)(442.72275546,305.97024242)
\lineto(451.36345647,298.46007859)
\curveto(451.84592868,297.91903366)(451.75657663,297.35370618)(451.09540005,296.76409445)
\curveto(450.73475547,296.44236583)(450.15115177,296.46145812)(449.34458721,296.82137139)
\curveto(444.24040751,299.33769029)(438.95955282,300.50940468)(433.50200731,300.33651806)
\curveto(428.07129747,300.13341256)(423.5076405,298.38344727)(419.81102272,295.08661694)
\curveto(416.17457099,291.84324005)(414.40302059,288.29370728)(414.49636622,284.43800796)
\curveto(414.61983496,280.60897926)(415.78052118,277.46230218)(417.97842838,274.99796729)
\curveto(420.65887247,271.99255361)(423.71618264,270.59139428)(427.15036806,270.7944851)
\curveto(430.58459772,270.99746969)(436.34023717,272.89322553)(444.41730367,276.48175832)
\curveto(450.33215545,279.16696417)(455.03470721,280.85194817)(458.52497306,281.5367154)
\curveto(462.04527395,282.24815903)(465.67238886,282.08365078)(469.40632867,281.04319015)
\curveto(473.1402282,280.00260279)(476.25357999,278.08485008)(478.74639336,275.28992628)
\curveto(482.9814161,270.54140563)(484.80569179,264.533156)(484.21922591,257.26515938)
\curveto(483.63269012,249.99708359)(480.48435117,243.81667333)(474.77419959,238.72391007)
\curveto(469.63501472,234.14038314)(463.03178094,231.13793879)(454.96447845,229.71656799)
\curveto(454.69316718,227.96371582)(454.916525,226.38225741)(455.63455259,224.97218801)
\curveto(456.25184981,223.79603366)(456.29001953,222.96672298)(455.74906187,222.48425348)
\curveto(455.26819837,222.05538263)(454.37309777,222.36323321)(453.06375737,223.40780615)
\curveto(451.75440939,224.45236763)(446.11863189,229.89409911)(436.15640798,239.73301691)
\curveto(435.95498582,240.20085412)(436.10972347,240.66261885)(436.6206214,241.11831247)
\curveto(437.0714375,241.52035094)(437.72692567,241.51141128)(438.58708787,241.09149345)
\curveto(453.58854486,233.58857664)(465.23665979,233.53609959)(473.53146759,240.93406216)
\curveto(477.64878154,244.60620444)(479.76554359,248.70644223)(479.88176011,253.23478785)
\curveto(480.00116484,257.81992005)(478.62686166,261.72036958)(475.75884646,264.93614814)
\curveto(473.7753145,267.16006459)(471.32150897,268.64082022)(468.39742249,269.37841949)
\curveto(465.50335133,270.14271949)(462.51019893,270.17117063)(459.41795633,269.463773)
\curveto(456.355747,268.78307593)(451.9707867,267.16550656)(446.26306227,264.61106005)
\curveto(438.91783639,261.35149618)(433.48466255,259.60844152)(429.96352445,259.38189083)
\curveto(426.4456744,259.21211442)(423.29454235,259.72020218)(420.51011886,260.90615564)
\curveto(417.69894876,262.1220762)(415.43561723,263.69177213)(413.72011749,265.61524814)
\curveto(410.15520403,269.61232415)(408.79553074,274.76689233)(409.64109356,281.07896813)
\curveto(410.5167892,287.41772126)(413.58430402,292.93249072)(418.84364722,297.62329305)
\curveto(420.82721284,299.39228555)(423.18235207,300.92620796)(425.90907197,302.22506489)
\curveto(428.66589216,303.55055685)(431.80769509,304.62595654)(435.33449018,305.45126718)
}
}
{
\newrgbcolor{curcolor}{0 0 0}
\pscustom[linestyle=none,fillstyle=solid,fillcolor=curcolor]
{
\newpath
\moveto(505.14541248,324.73527346)
\curveto(512.04876985,328.24815253)(518.06229795,329.91525721)(523.18601482,329.7365925)
\curveto(528.33973372,329.58459369)(532.60526612,327.61525434)(535.98262481,323.82856854)
\curveto(539.60114351,319.77127425)(541.21017754,314.48836822)(540.80973171,307.97983463)
\curveto(540.43600322,301.44115274)(536.89818833,295.18317521)(530.19627644,289.20588327)
\curveto(528.93399399,288.08008131)(526.85339491,286.35933864)(523.95447296,284.04365012)
\curveto(522.10087379,282.55234383)(521.09895356,281.73969208)(520.94870928,281.60569243)
\curveto(511.69218816,273.35000004)(505.52145933,267.68458073)(502.43650425,264.60941753)
\curveto(501.35457094,263.64446493)(500.55896758,263.44749948)(500.04969178,264.01852059)
\curveto(499.78164787,264.31905282)(499.74428679,264.66344841)(499.93760844,265.05170837)
\curveto(500.1609786,265.4667636)(501.11863456,266.7525531)(502.81057918,268.90908073)
\curveto(503.82508743,270.1916125)(504.31610451,271.54685311)(504.28363188,272.97480662)
\curveto(504.25114552,274.40273543)(503.30285585,276.52475366)(501.43876001,279.3408677)
\curveto(499.57465613,282.15694429)(496.44466772,286.02938415)(492.04878536,290.95819892)
\curveto(487.706522,295.82684588)(484.26869043,299.80244418)(481.73528034,302.88500575)
\curveto(479.17509748,305.99753635)(477.18468882,308.35024222)(475.76404839,309.94313041)
\curveto(471.34138893,314.90191908)(467.59449381,318.70978942)(464.5233518,321.36675286)
\curveto(461.48232914,324.05039893)(459.25432965,325.48972798)(457.83934665,325.68474432)
\curveto(456.39763869,325.90968409)(454.4624407,325.23593457)(452.0337469,323.66349373)
\curveto(451.05502243,323.00632522)(450.33780241,322.93322779)(449.8820847,323.44420119)
\curveto(449.39965794,323.98510251)(449.59419746,324.64424444)(450.46570385,325.42162895)
\lineto(458.7909536,332.60392203)
\curveto(461.46575939,334.98941743)(463.92162369,337.28766924)(466.15855388,339.49868435)
\curveto(471.17672161,344.45980994)(474.45215571,347.62391743)(475.98486602,348.99101633)
\curveto(481.54479672,353.94969163)(487.04823774,356.67272911)(492.49520561,357.16013695)
\curveto(497.91212976,357.62054758)(502.49687833,355.74705171)(506.24946508,351.53964373)
\curveto(509.19791032,348.23366082)(510.6639989,344.44205252)(510.64773523,340.16480746)
\curveto(510.63144509,335.88740331)(508.79733934,330.74423045)(505.14541248,324.73527346)
\moveto(467.5694109,335.73875058)
\curveto(473.16906269,328.91570788)(477.79155437,323.46058178)(481.43689979,319.37335589)
\lineto(487.43250253,312.74166433)
\curveto(488.23663636,311.8400053)(488.86612904,311.37620411)(489.32098248,311.35025935)
\curveto(489.80590567,311.35101727)(490.61938092,311.86070002)(491.76141068,312.87930914)
\curveto(497.47159342,317.97204119)(501.13001945,323.09651864)(502.73669975,328.25275688)
\curveto(504.31656528,333.43891341)(502.90856222,338.4964177)(498.51268633,343.42528493)
\curveto(494.78691145,347.60264746)(490.72038097,349.56061379)(486.3130827,349.29918979)
\curveto(481.90905608,349.09444612)(477.79863592,347.2900565)(473.98180992,343.88601552)
\curveto(471.75786965,341.90242678)(469.62040545,339.18667452)(467.5694109,335.73875058)
\moveto(491.53232524,308.87083788)
\curveto(491.78168444,307.74419275)(492.73728906,306.24923384)(494.39914197,304.38595666)
\lineto(506.5475112,290.12959201)
\curveto(510.94337905,285.20077807)(513.51617101,282.55809352)(514.2658948,282.20153041)
\curveto(515.04563845,281.87172828)(516.34605784,282.16819097)(518.16715687,283.09091938)
\curveto(520.01826796,284.04040698)(522.55169954,285.94918286)(525.76745919,288.81725275)
\curveto(528.62251938,291.36361671)(530.58819011,293.81823089)(531.66447728,296.18110265)
\curveto(532.74394539,298.60076098)(532.98967467,301.6528001)(532.40166585,305.33722918)
\curveto(531.81358707,309.02156882)(530.05874019,312.50168021)(527.13711997,315.7775738)
\curveto(522.79482152,320.64619593)(518.16255924,322.93597186)(513.24031924,322.64690843)
\curveto(508.34810115,322.38452508)(503.09199183,319.74717846)(497.4719755,314.73486065)
\curveto(495.57860094,313.04614647)(493.5987195,311.09147417)(491.53232524,308.87083788)
}
}
{
\newrgbcolor{curcolor}{0 0 0}
\pscustom[linestyle=none,fillstyle=solid,fillcolor=curcolor]
{
\newpath
\moveto(504.74371052,373.10256229)
\lineto(527.32898612,393.24588459)
\curveto(528.62131462,394.39837867)(529.50870339,394.70418535)(529.99115509,394.16330554)
\curveto(530.47365199,393.62225706)(530.27423832,392.8778286)(529.39291347,391.93001791)
\curveto(527.44676255,389.81646779)(526.39163096,387.82320658)(526.22751553,385.95022831)
\curveto(526.06345283,384.07709781)(526.7964934,381.68216414)(528.42663945,378.76542013)
\curveto(530.08688741,375.87534237)(534.48194831,370.433214)(541.61183536,362.4390187)
\curveto(550.16236316,352.85187931)(556.35655751,346.45127892)(560.19443699,343.23719831)
\curveto(564.06235484,340.04985501)(568.43226606,338.65924521)(573.30418378,339.06536471)
\curveto(578.17930951,339.52828344)(583.14138207,342.01130417)(588.19041634,346.51443435)
\curveto(593.47980794,351.23192418)(596.60904717,355.8304717)(597.57814343,360.31009071)
\curveto(598.54714972,364.78960876)(597.66832602,369.37478395)(594.94166969,374.06563003)
\curveto(592.24498224,378.78315177)(584.98634947,387.76875552)(573.16574962,401.02246825)
\curveto(567.08119201,407.84453598)(562.39452026,411.67759317)(559.10572033,412.52165129)
\curveto(555.79010325,413.3955576)(552.74983404,412.59957222)(549.98490358,410.13369275)
\curveto(548.87292434,409.14183971)(548.03549007,408.96152576)(547.47259826,409.59275036)
\curveto(546.93652082,410.19372187)(547.40479185,411.15095864)(548.87741275,412.46446354)
\lineto(566.90956892,428.54695639)
\curveto(567.72100082,429.27054913)(568.39476521,429.33186922)(568.9308641,428.73091685)
\curveto(569.30610556,428.31004806)(569.20335145,427.75974736)(568.62260146,427.08001311)
\curveto(566.63336001,424.71219294)(565.59365952,422.48987624)(565.50349687,420.41305637)
\curveto(565.44336197,418.36281946)(566.54800658,415.73273575)(568.81743402,412.52279737)
\curveto(571.08682901,409.31265361)(574.74112264,404.8825981)(579.78032588,399.23261755)
\curveto(586.07926876,392.16994071)(591.15621021,385.93301763)(595.01116547,380.52182959)
\curveto(598.89608945,375.1373043)(601.05258649,370.45058796)(601.48066306,366.46166652)
\curveto(601.88183772,362.50268539)(600.99441922,358.20384256)(598.81840491,353.56512515)
\curveto(596.64554758,348.98317796)(593.00458912,344.41387535)(587.89551862,339.85720363)
\curveto(582.66616326,335.19325534)(577.46609555,332.03931534)(572.29529989,330.39537416)
\curveto(567.12771427,328.80824955)(562.66316509,328.79244385)(558.90163896,330.34795702)
\curveto(555.14334965,331.96028685)(550.8250325,335.50133631)(545.94667457,340.97111602)
\lineto(534.98986983,353.61920909)
\curveto(525.09916423,364.70893026)(518.57397636,371.0571428)(515.41428665,372.66386575)
\curveto(512.25792074,374.32732073)(509.09012976,373.79531854)(505.91090421,371.0678576)
\curveto(504.88912974,370.15645601)(504.15038407,369.95624113)(503.69466499,370.46721234)
\curveto(503.13182721,371.09827654)(503.4815087,371.97672565)(504.74371052,373.10256229)
}
}
{
\newrgbcolor{curcolor}{0 0 0}
\pscustom[linestyle=none,fillstyle=solid,fillcolor=curcolor]
{
\newpath
\moveto(633.26386207,436.00660709)
\curveto(639.01949099,437.90229931)(643.59329315,438.58212109)(646.98528228,438.04607446)
\curveto(650.35041878,437.53995008)(654.25977825,435.57670528)(658.7133724,432.15633416)
\curveto(664.55990062,427.65797331)(669.21571759,425.28128613)(672.68083727,425.0262655)
\curveto(676.11905295,424.80119316)(679.34939185,425.92857639)(682.37186366,428.40841856)
\curveto(683.2368585,429.07196687)(683.91062289,429.13328697)(684.39315884,428.59237902)
\curveto(684.87557149,428.05135868)(684.65097841,427.36541262)(683.71937892,426.5345388)
\curveto(675.30432014,419.02932883)(669.95397718,414.74310915)(667.66833399,413.6758669)
\curveto(665.38585039,412.66539978)(661.31607047,414.56650841)(655.45898205,419.37919852)
\curveto(649.79595596,424.09513508)(645.40128451,427.29828782)(642.27495452,428.98866636)
\curveto(639.15183489,430.73578939)(636.02385321,431.39942344)(632.8910001,430.97957051)
\curveto(629.76137613,430.6164581)(626.13790011,428.59884842)(622.02056118,424.92673543)
\curveto(620.4577766,423.53286824)(618.83203905,421.78613009)(617.14334366,419.68651574)
\curveto(618.6817415,417.6590733)(620.36227798,415.62355457)(622.18495812,413.57995343)
\curveto(629.07361956,405.85614641)(633.64169433,400.94604252)(635.88919615,398.84962704)
\curveto(638.10986754,396.78321221)(640.21521942,395.69315291)(642.20525808,395.57944585)
\curveto(644.19850726,395.5225501)(646.46916859,396.57643157)(649.0172489,398.74109341)
\curveto(650.12595504,399.67596211)(650.89475427,399.90298112)(651.3236489,399.42215113)
\curveto(651.83289875,398.8511064)(651.42635957,397.97590673)(650.10403013,396.7965495)
\lineto(627.78923687,376.8944646)
\curveto(626.49692859,375.7418826)(625.59613776,375.4511027)(625.08686167,376.02212404)
\curveto(624.63118915,376.53303099)(624.91426464,377.24415645)(625.936089,378.15550257)
\curveto(628.22015664,380.19261053)(629.48972125,381.94544322)(629.74478666,383.41400592)
\curveto(630.02989162,384.9093507)(629.78460382,386.60614921)(629.00892253,388.50440655)
\curveto(628.20642457,390.43268523)(626.1031166,393.30523391)(622.69899231,397.12206121)
\lineto(614.40676018,406.23804839)
\lineto(601.94770238,420.29825605)
\curveto(595.56834072,427.45095627)(591.18139597,432.03696983)(588.78685496,434.05631048)
\curveto(586.42243543,436.10233193)(584.48603191,437.15421409)(582.97763861,437.21196011)
\curveto(581.4425207,437.29963005)(579.00128629,436.25549689)(575.65392805,434.07955749)
\curveto(574.94243792,433.60676981)(574.39904042,433.58078175)(574.02373391,434.00159325)
\curveto(573.59491658,434.48238866)(574.01160839,435.28570385)(575.27381059,436.41154125)
\lineto(601.43872071,460.79967574)
\curveto(608.80186508,467.36659868)(615.39616663,470.7118139)(621.22164513,470.83533145)
\curveto(627.07718023,470.98545623)(631.82762772,469.01692552)(635.47300186,464.92973341)
\curveto(637.77814413,462.34503441)(639.20606299,459.38273679)(639.75676273,456.04283167)
\curveto(640.337482,452.72956338)(640.1400825,449.77458814)(639.16456365,447.17789708)
\curveto(638.18901194,444.58105646)(636.22211338,440.85729685)(633.26386207,436.00660709)
\moveto(615.13303206,421.94053526)
\curveto(620.79610736,427.20709253)(624.76196478,431.47261006)(627.03061623,434.73710065)
\curveto(629.2992591,438.00146655)(630.39501607,441.70370794)(630.31789043,445.84383592)
\curveto(630.2708032,450.01062113)(628.81324252,453.70191611)(625.94520404,456.91773194)
\curveto(622.91633437,460.31370009)(619.14911914,462.26882131)(614.64354703,462.78310144)
\curveto(610.13799111,463.2972008)(605.40579011,461.34295511)(600.44692982,456.92035853)
\curveto(597.7421314,454.50790249)(595.07345136,451.48024899)(592.4408817,447.83738897)
\curveto(599.45140442,439.250934)(605.14118706,432.50837882)(609.51024669,427.60970317)
\curveto(611.01129108,425.92664839)(612.885551,424.03692765)(615.13303206,421.94053526)
}
}
{
\newrgbcolor{curcolor}{0 0 0}
\pscustom[linestyle=none,fillstyle=solid,fillcolor=curcolor]
{
\newpath
\moveto(706.95581738,519.45662243)
\curveto(706.5650958,519.10807429)(705.73050664,519.22707558)(704.45204741,519.81362665)
\curveto(697.16288942,523.02523477)(690.17630399,524.29443222)(683.49227015,523.62122282)
\curveto(676.83828631,522.97462027)(671.22722445,520.61425458)(666.65906774,516.54011869)
\curveto(662.90238734,513.18951114)(660.28446301,508.80417625)(658.8052869,503.38410085)
\curveto(657.29935246,497.99392221)(657.55191804,491.93288783)(659.56298441,485.20097954)
\curveto(661.54729094,478.49900437)(664.92500052,472.47327974)(669.69612328,467.12378757)
\curveto(676.50438379,459.49014028)(684.67196071,454.9304968)(694.19887853,453.44484346)
\curveto(703.72577137,451.95913318)(711.91533028,454.27196286)(718.76757984,460.38333943)
\curveto(721.20188142,462.5544384)(722.94052571,464.77959277)(723.98351791,467.05880923)
\curveto(725.02642556,469.33794153)(725.6693426,471.85389067)(725.91227098,474.60666418)
\curveto(726.1851645,477.38614256)(726.16365371,479.25554319)(725.84773854,480.21487167)
\curveto(725.56178866,481.2008938)(724.10543352,483.1665571)(721.47866875,486.11186746)
\curveto(717.08275199,491.04059076)(713.52672778,493.69668481)(710.81058543,494.08015758)
\curveto(708.12443844,494.49029628)(705.33880845,493.40880205)(702.45368709,490.83567162)
\lineto(701.09640122,489.54419803)
\curveto(700.22482849,488.76681376)(699.5344126,488.66366276)(699.02515149,489.23474472)
\curveto(698.40863923,489.92591243)(698.58124877,490.70039445)(699.54298064,491.55819313)
\lineto(702.0126538,493.59896265)
\lineto(712.08020334,502.82081228)
\lineto(724.3262341,514.22842092)
\curveto(725.01742344,514.84482662)(725.64448276,514.83751167)(726.20741395,514.20647606)
\curveto(726.77025614,513.575262)(726.68780728,512.88119235)(725.96006714,512.12426504)
\curveto(724.25752671,510.28200059)(723.10167748,508.52266902)(722.49251597,506.84626506)
\curveto(721.88327289,505.16969737)(721.80528644,503.80511277)(722.25855638,502.75250718)
\curveto(722.7117452,501.69974604)(723.79609199,500.21169032)(725.51159998,498.28833556)
\curveto(731.78372212,491.25572838)(735.06275796,487.24642275)(735.34871735,486.26040666)
\curveto(735.63456945,485.27426802)(734.24680475,481.20367203)(731.1854191,474.04860649)
\curveto(728.12393449,466.89344282)(723.93347655,460.9437202)(718.6140327,456.19942077)
\curveto(709.80830167,448.34578006)(699.76029597,444.45638099)(688.46998546,444.53121188)
\curveto(677.17966142,444.60600366)(667.67470836,448.97112249)(659.95509776,457.62658144)
\curveto(654.64790217,463.57715438)(651.32258789,470.27014453)(649.97914492,477.70557196)
\curveto(648.63901484,485.19775164)(649.35913199,492.55797757)(652.13949854,499.78627185)
\curveto(654.89312008,507.04447008)(658.92965645,513.04577192)(664.24911975,517.79019538)
\curveto(669.23802875,522.23958781)(675.89081271,525.60998244)(684.20749157,527.90138939)
\curveto(686.55409296,528.53726316)(688.0579841,529.15009691)(688.71916951,529.73989249)
\curveto(689.65082297,530.5707157)(690.01025277,531.61973787)(689.79745998,532.88696217)
\curveto(689.6317604,533.98014815)(689.72923604,534.68761986)(690.08988719,535.00937942)
\curveto(690.78111076,535.62576433)(691.96415358,535.11606672)(693.63901919,533.48028504)
\curveto(696.25850428,530.90612703)(700.42005474,527.11734155)(706.12368305,522.11391723)
\curveto(706.45180703,521.86689272)(706.72309775,521.62321478)(706.937556,521.38288267)
\curveto(707.58082984,520.66150074)(707.58691696,520.01941464)(706.95581738,519.45662243)
}
}
\end{pspicture}
}} % Logo-skalieren und einfügen
\fancyhead[C]{}
\fancyhead[R]{Kaiser, Baß}
%-------Fußzeile:-------
\fancyfoot[L]{}
\fancyfoot[C]{\thepage}
\fancyfoot[R]{}
%--------------------------------------------------
